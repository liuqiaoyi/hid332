\documentclass[sigconf]{acmart}

\usepackage{hyperref}

\usepackage{endfloat}
\renewcommand{\efloatseparator}{\mbox{}} % no new page between figures

\usepackage{booktabs} % For formal tables

\settopmatter{printacmref=false} % Removes citation information below abstract
\renewcommand\footnotetextcopyrightpermission[1]{} % removes footnote with conference information in first column
\pagestyle{plain} % removes running headers

\begin{document}
\title{Big Data Analytics in Agriculture}


\author{Judy Phillips}
\orcid{xxxx-xxxx-xxxx}
\affiliation{%
  \institution{Indiana University}
  \streetaddress{3209 E 10th St}
  \city{Bloomington} 
  \state{Indiana} 
  \postcode{47408}
}
\email{judkphil@iu.edu}

% The default list of authors is too long for headers}
\renewcommand{\shortauthors}{B. Trovato et al.}


\begin{abstract}
This paper discusses ways that Big Data analytics and Data Science is improving the industry of Agriculture.
\end{abstract}

\keywords{Precision farming, Smart farming }


\maketitle

\section{Introduction}

Big Data Analytics is revolutionizing the Agricultural Industry. Farmers are using the information that is being made available with big data and the Internet things to increase efficiency, optimize results, and reserve resources. Internet connected devices are becoming common place on farms. Almost all new farm equipment has sensors. Sixty percent of farmers report some type of internet sourced data to make operational decisions. The related software market is growing rapidly. In 2010 the investment in Agricultural Technology was $500 million. In 2015 the investment had grown to $4.2 billion.  In this paper will discuss ways in how Big Data is impacting the field of Agriculture with Smart or Precision farming.

The \textit{proceedings} are the \cite{VanGundy09}

\section{THE SMART FARM AND PRECISION AGRICULTURE}

Precision agriculture is a specific farm management technique that uses sensor and analytic technology to measure, observe and respond to crop and livestock management in real time.
As the name implies, Precision farming matches farming techniques to the specific crop and livestock needs. The objective of precision farming is to ensure that crops receive that exact inputs that need, at the correct time, and in precise amounts. Examples of crop inputs include: water, fertilizer, herbicides and pesticides. This strategy enables a farmer to get the most productivity out each and every resource. Solutions are customized to each individual farmer’s unique needs.

Processes that are managed Precision techniques include: seeding, planting, harvesting, weed control, fertilizer management, breeding, disease control, pesticide management, light and energy management. Benefits of precision farming include improved yields, improved crop quality, reduced costs, and increased profits.  Because fertilizers, pesticides and weed control applications are not being overused, precision farming also helps the environment.

The life cycle of the smart farming process is as follows: Sensoring and monitoring, analysis and decision making, and intervention.  We will discuss each of this aspects in more detail below.

Sensor technologies measure and monitor data.  Sensors register and report deviations in real time. Sensors include devices that are located locally on the farm and external satellites. 

Types of local sensors include: connected farming equipment (tractors, harvesters), chips planted into livestock, and drones. Examples of the types of data that may be collected via local sensors include: Rainfall and water measurements, crop health, livestock health, weather information, yield monitoring, and lighting and energy management. Drones can collect aerial images of fields. Data is oftentimes collected in very precise detail. Data can be to the square meter of land and even in at the individual plant level. 

Data collected with local sensors is often supplemented with information other external sources such as satellites and the cloud. Data that may be collected via satellite and available in real time on the cloud includes: Weather and climate data (historical and real time), soil type analysis, market information, and livestock movements. Data from collected from orbiting satellites can also be very granular and personalized. For example soil characteristics such as texture, organic matter, and fertility is collected to the meter at locations throughout the world. 

After the data is collected it must me consolidated and analyzed. A significant amount of this support is being provided by Machine supplier companies that have been servicing the farming industry for generations: John Deere, Dupont Pioneer, and Monsanto to name a few. Most of this support is in the form of software decision support tools. Now, in addition to selling seeds and machinery, these companies are developing software tools and selling data science services. Data is collected from each individual farmer, this data in combined with data available from other sources, statistical models and algorithyms are applied, and this results in custom solutions for each farmer. The objective of the recommendations are to increase yields. Recommendations include how to manage the soils nitrogen content and recommendations on how far apart to place seeds based upon the field position. Dupont Pioneer has mapped and has data on 20 million acres in the United States. Another company, Cropin supports farmers worldwide, including farmers in extremely remote rural areas. This company has mapped over one billion acres globally and can provide data by individual farm, farm clusters, districts, states, and even countries (India). This software development field is expected to grow by 14% by 2022. 

Typical Storage vehicles for this data are typically cloud based. 

After the data is analyzed it is downloaded from the cloud and made available to the farmers in various ways via wireless technology:
It may be downloaded to an Ipad or computer in a tractor. Other information can be sent to Smart phones. By interacting with the Internet of Things farmers can manage operational activities from anywhere in the world. Other devices are self automated. One such self automated technology is Variable rate technology (VRT). Variable rate technology is built into equipment such as irrigation systems, feeders, and milking devices. These devices automatically operate in such a way as to deliver optimal results.

\section {Examples}

In this section I will share some examples of how Big Data Analytics is enhancing Agricultural productivity.

In this section I will give some detailed examples of Big Data Analytics in Agricultural science:

Following are some examples of technology in the world of crop science: Satellite systems and sensors can monitor the development of crops in detail. Individual plants can be monitored for nutrients, growth rate and health. In this way disease outbreaks can be recognized and addressed immediately.  Entire fields can be mapped with GPS coordinates to collect data concerning soil conditions and elevation. Algorithyms instruct the tractor's planting mechansim where to place every seed. This same technology can even tell if a single seed has been missed. GPS units on tractors, combines, and trucks help determine the optimal usage of equipment.

Big Data technology also improves the field of Animal and livestock management. Milk cows are tagged with chips that monitor the health of the animal. Milking machines shut down when the animal is sick. Sensors indicate when livestock are ready to inseminate or give birth. 

Data and consolidated can offer insights and information that has never before been possible. Big data companies can test and gather information about the effectiveness of different kinds of seeds across many different conditions, soil types, and climates. The origin of crop diseases can be identified quickly and efficiently with web searches similar to the way that flu epidemics are currently identified. This will enable players to take corrective action quickly. Historical analytics can determine the best crops to plant. 





\section{Challenges and Issues}

Machine suppliers in the form of big companies have played a big role in this evolution by developing desision support tools that provide information to better manage farms. When farmers share their data with big companies such as John Deere, there is some concern that these big players have the potential to consoldiate the data and use it to manipulate the market. There need to be defined standards for the use of such data. The American Farm Bureau association is seeking data usage assurance regulations within the industry.

\section{Potential Worldwide Impact}

Improvements to agricultural productivity as result of big data technology are beyond substantial. Big data is being referred to as the most major revolution productivity in farming since mechanization.  In 2009, the United Nations estimated that 900 people in the world were undernourished and that 65 countries face alarming food shortages. Big Data is expected to make an impact on Food Insecurity throughout the world as farmers throughout the world adopt these techniques. Will enable even small holder farmers to make full use of their productive potential. The use of precision farming techniques and digital technologies will enable farmers to maximize the use of every inch of soil and even the production of each individual plant.

\section{Conclusion}
To summarize the availablity of Big Data Analytics is changing the Agricultue farming operation and management. Combinaion of access to real time data, data analytics and information that has never before bee available.
automation of this data
All of this leads to better farm management because of access to information that was previously impossible to get


\begin{acks}

  .

\end{acks}

\bibliographystyle{ACM-Reference-Format}
\bibliography{report} 

\end{document}
\documentclass{article}
\usepackage[utf8]{inputenc}

\title{paper1}
\author{judkphil }
\date{September 2017}

\begin{document}

\maketitle

\section{Introduction}

\end{document}






\documentclass[sigconf]{acmart}

\usepackage{hyperref}

\usepackage{endfloat}
\renewcommand{\efloatseparator}{\mbox{}} % no new page between figures

\usepackage{booktabs} % For formal tables

\settopmatter{printacmref=false} % Removes citation information below abstract
\renewcommand\footnotetextcopyrightpermission[1]{} % removes footnote with conference information in first column
\pagestyle{plain} % removes running headers

\begin{document}
\title{Big Data Analytics in Agriculture}


\author{Judy Phillips}
\orcid{xxxx-xxxx-xxxx}
\affiliation{%
  \institution{Indiana University Bloomington}
  \streetaddress{3209 E 10th St}
  \city{Bloomington} 
  \state{Indiana} 
  \postcode{47408}
}
\email{judkphil@iu.edu}

% The default list of authors is too long for headers}
\renewcommand{\shortauthors}{B. Trovato et al.}


\begin{abstract}
This paper discusses ways that Big Data analytics and Data Science is improving the industry of Agriculture.
\end{abstract}

\keywords{Precision farming, Smart farming }


\maketitle

\section{Introduction}

Big Data Analytics is revolutionizing the Agricultural Industry. Farmers are using the information that is being made available with big data and the Internet things to increase efficiency, optimize results, and reserve resources. Internet connected devices are becoming common place on farms.  Almost all new farm equipment has sensors. Sixty percent of farmers report some type of internet sourced data to make operational decisions. The related software market is growing rapidly. In 2010 the investment in Agricultural Technology was $500 million. In 2015 the investment had grown to $4.2 billion.  In this paper will discuss ways in how Big Data is impacting the field of Agriculture with Smart or Precision farming.

The \textit{proceedings} are the \cite{VanGundy09}

\section{THE SMART FARM AND PRECISION AGRICULTURE}

Precision agriculture is a specific farm management technique that uses sensor and analytic technology to measure, observe and respond to crop and livestock management in real time.
As the name implies, Precision farming matches farming techniques to the specific crop and livestock needs. The objective of precision farming is to ensure that crops receive that exact inputs that need, at the correct time, and in precise amounts. Examples of crop inputs include: water, fertilizer, herbicides and pesticides. This strategy enables a farmer to get the most productivity out each and every resource. Solutions are customized to each individual farmer’s unique needs.

Processes that are managed Precision techniques include: seeding, planting, harvesting, weed control, fertilizer management, breeding, disease control, pesticide management, light and energy management. Benefits of precision farming include improved yields, improved crop quality, reduced costs, and increased profits.  Because fertilizers, pesticides and weed control applications are not being overused, precision farming also helps the environment. 


The lifecycle of the smart farming process is as follows: Sensoring and monitoring, analysis and decision making, and intervention.  We will discuss each of this aspects in more detail below.

Sensor technologies measure and monitor data.  Sensors register and report deviations in real time. Sensors include devices that are located locally on the farm and external satellites. 

Types of local sensors include: connected farming equipment (tractors, harvesters), chips planted into livestock, and drones. Examples of the types of data that may be collected via local sensors include: Rainfall and water measurements, crop health, livestock health, weather information, yield monitoring, and lighting and energy management. Data is oftentimes collected in very precise detail. Data can be to the square meter of land and even in at the individual plant level. 

Data collected with local sensors is often supplemented with information other external sources such as satellites and the cloud. Data that may be collected via satellite and available in real time on the cloud includes: Weather and climate data (historical and real time), soil type analysis, market information, and livestock movements. Data from collected from orbiting satellites can also be very granular and personalized. For example soil characteristics such as texture, organic matter, and fertility is collected to the meter at locations throughout the world. 

\section{Challenges and Issues}

Machine suppliers in the form of big companies have played a big role in this evolution by developing desision support tools that provide information to better manage farms. When farmers share their data with big companies such as John Deere, there is some concern that these big players have the potential to consoldiate the data and use it to manipulate the market. There need to be defined standards for the use of such data. The American Farm Bureau association is seeking data usage assurance regulations within the industry.

\section{Conclusions -Potential Worldwide Impact}

Improvements to agricultural productivity as result of big data technology are beyond substantial. Big data is being referred to as the most major revolution productivity in farming since mechanization.  In 2009, the United Nations estimated that 900 people in the world were undernourished and that 65 countries face alarming food shortages. Big Data is expected to make an impact on Food Insecurity throughout the world as farmers throughout the world adopt these techniques. Will enable even small holder farmers to make full use of their productive potential. The use of precision farming techniques and digital technologies will enable farmers to maximize the use of every inch of soil and even the production of each individual plant.


\begin{acks}

  The authors would like to thank Dr. Yuhua Li for providing the
  matlab code of the \textit{BEPS} method.

\end{acks}

\bibliographystyle{ACM-Reference-Format}
\bibliography{report} 

\end{document}

*********************************************************************************************************************
Notes and Drafts - Do not use:
\documentclass[sigconf]{article}
\usepackage{hyperref}

\usepackage{endfloat}
\renewcommand{\efloatseparator}{\mbox{}} % no new page between figures

\usepackage{booktabs} % For formal tables

\settopmatter{printacmref=false} % Removes Citation information below abstract

\renewcommand\footnotetextcopyrightpermission[1]{} % removes footnote with conference information in first column

\pagestyle{plain} % removes running headers

\begin{document} 
\title{Big Data Analytics in Agriculture}

\author{Judy Phillips }
\affilition{%
  \institution{Indiana University Bloomington}
  \streetaddress{Smith Reasearch Center}
  \city{Bloomington}
  \state{IN}
  \postcode{47408}
  \country{USA}

}
\email(judkphil@iu.edu}


\begin{abstract}
This paper discusses how big data technologies are changing and improving the field of agriculture    
    
\end{abstract}

\keywords (Big Data, Precision farming}

\begin{document}

\maketitle

\section{Introduction}
Big Data Analytics is starting to make significant impacts to the Agricultural Industry. Farmers are utilize data from data various sources to manage operational decisions. Internet connected devices are becoming common place on farms.  Almost all new farm equipment has sensors. Sixty percent of farmers report some type of internet sourced data to make operational decisions. The related software market is growing rapidly. In 2010 the investment in Agricultural Technology was $500 million. In 2015 the investment had grown to $4.2 billion.  In this paper will will discuss ways in which Big Data and Data science is improving the field of Agriculture with precision farming.

\section{Global food insecurity}

\section{History}
To date, there have been two significant stages of agricultural revolution. The first phase, which lasted through the 1920's was the pre-industrial phase. Farming at this time was very labor intensive. One acre could feed 1/2 of a person. The next phase was earmarked with the implementation mechaninzation (tractors, harvesters), plant science, and chemical fertilizers and pesticides.  This phase lasted between 1920 until 2010.  Production had improved significantly - One acre was feeding 5 people. Beginning around 2011 we started to enter the third phase the third phase with the implementation of Big Data Analytics. This phase is the era of Big data analytics and the Internet of Things. Data is now being aggregated from diverse technological sources including  locally placed , and satellites. This information is being analyzed in the cloud back smart devices where it can be used to manage farming operational activitiesin real time. 

It is expected that the digital revolution and Big Data Analytics will play a significant role in meeting the growing food production needs of the future. It is estimated that the world population will be 9.6 billion people in 2050. Food production must effectively double fom what we currently produce in order to feed everyone. The use of precision farming techniques and digital technologies will enable farmers to maximize the use of every inch of soil and even the production of each individual plant

\section{PRECISION FARMING}

Preciesion agriculture is a specific farm management technique that uses sensor and analytic technology to measure, observe and respond to crop and live stock management in real time.
As the name implies, Precision farming matches farming techniques to the specific crop and livestock needs. the objective of precision farming is to ensure that crops receive that exact inputs that need, at the correct time, and in precise amounts.Examples of crop inputs include: water, fertilizer, herbicides and pesticides.

Benefits of precision farming include improved yields, improved crop quality, reduced costs, and increased profits.

Precision farming also helps the environment. Because fertilizers, pesticides and weed control applications are not being overused, precision farming also helps the enviorment. 




Vast amounts of data information is collected from smart machines in the field, satellites and other cloud resources. The data is crunched and analyzed by smart devices and turned into knowledge that can be used to make more efficient operational farming decisions. 


Sensor technologies measure and montitor data. Examples of sensors include: connected farming equipment (tractors, harvesters), drones, and chips planted into livestock. Sensors register and report deviations in real time. External data from the cloud, such as weather information is used to complement the information. 

Examples of the types of data that is collected via local sensors include: Rainfall and water measurements, crop health, livestock health, weather information, yield monitoring, lighting and energy management. Data can be collected at the individual plant level. Data that may be collected via sattelite and available in real time on the cloud includes: Weather and climate data (historical and real time), soil type analyis. market information, livestock movements. This data is at the detail of each square inch of land or each individual plant.

Data is used to manage processes such as: seeding, planting, harvesting, weed control, fertilzer management, breeding, disease control, pesticide management, light and energy management. 

Internet of things
/section{Examples}

* Sensors on fields and crops at granular data points collect data on local soil conditions, wind fertilzer requirement, water availability, and pest control
*GPS Units of tractors, combines, and trucks help determine optimal usage of equipment.
*Individual plants monitored for nutrients and growth rates
*Historical analysis can determine best crops to plant
*Entire fields mapped with GPS coordinates monitoring soil and elevation. Algorithyms tell tractos planting mechanism where to plant every seed. Can even identify if a seed has been missed. 
*Sensors indicate when livestock are ready to give birth or inseminate. 

/section{challenges}

Machine suppliers in the form of big companies have played a big role in this evolution by developing desision support tools that provide information to better manage farms. When farmers share their data with big companies such as John Deere, there is some concern that these big players have the potential to consoldiate the data and use it to manipulate the market. There need to be defined standards for the use of such data. The American Farm Bureau association is seeking data usage assurance regulations within the industry.

section/{conclusion}

Helps farmers - Increases efficiency, optimizes results, reserves resources,
improves profitablity.

Protects enviroment

Global food security for world






\bibliographystyle{plain}
\bibliography{references}
\end{document}
